\documentclass{article}
\usepackage{cuisine}

\title{Das Kochbuch um die Welt zu retten}
\author{Julius \& Anna}
\date{\today}

\begin{document}

\maketitle
%\tableofcontents
%Wir sollten vielleicht das Rezept noch zitieren wegen Plagiats
\begin{recipe}{Karottencremesuppe}{4 Portionen}{mindestens 25 Minuten}
  \ingredient[1]{Esslöffel}{Kokos- oder Olivenöl}
  \ingredient[1]{}{Zwiebel, gewürfelt}
  \ingredient[3]{}{Knoblauchzehen, gehackt}
  \ingredient[25]{g}{Ingwer, gehackt}
  \ingredient[2-3]{Teelöffel}{gelbes Currypulver}
  \ingredient[500]{g}{Karotten, geschält und gewürfelt}
  \ingredient[120]{g}{Kartoffeln, geschält und gewürfelt}
  \ingredient[120]{ml}{Kokosmilch (vollfett aus der Dose)}
  \ingredient[\fr{1}{4}]{Teelöffel}{Chillipulver}
  \ingredient[600]{ml}{Gemüsebrühe}
  \ingredient{nach Bedarf}{Salz und Pfeffer}
  \ingredient[1]{Spritzer}{Zitronensaft}
\begin{enumerate}
  \item Das Öl in einem großen Topf bei mittlerer Hitze erhitzen. Zwiebel dazugeben und 2-3 Minuten glasig anbraten. Dann Knoblauch und Ingwer hinzugeben und eine weitere Minute unter Rühren anbraten. Nun das Currypulver und die Karotten hinzugeben und nochmals eine Minute weiterbraten.
  \item Anschließend mit der Gemüsebrühe ablöschen. Die Kartoffeln hinzugeben und alles zum Kochen bringen. Die Hitze reduzieren und etwa 15 Minuten abgedeckt köcheln lassen odere bis die Karotten und Kartoffeln weich genug zum Pürieren sind.
  \item Zuletzt die Kokosmilch hinzugeben und die Suppe im Mixer oder mit einem Pürierstab cremig pürieren.
  \item Anschließend nochmals zurück auf den Herd stellen, mit Salz, Pfeffer, Chilli und Zitronensaft abschmecken. Bei Bedarf nochmals erwärmen (zum Beispiel die Suppe noch mindestens 5 Minuten am ausgeschalteten Herd stehen lassen, damit sich die Aromen entfalten können)
  \item Suppe warm servieren und nach Wunsch mit Granatapfelkernen, Kürbiskernen, Sesam und frischen Kräutern garnieren.

\end{enumerate}
\end{recipe}
\begin{recipe}{Kichererbsen-Karfiol-Curry}{4 Portionen}{}
  \ingredient[1]{mittelgroßer}{Karfiol}
  \ingredient[1]{rote}{Zwiebel}
  \ingredient[2]{Esslöffel}{Öl}
  \ingredient[3]{Esslöffel}{Currypulver}
  \ingredient[400]{ml}{Kokosmilch}
  \ingredient[500]{g}{Kichererbsen, gekocht aus der Dose}
  \ingredient[1]{Teelöffel}{Salz}
  \ingredient[1]{Teelöffel}{Pfeffer}
  \ingredient[200]{ml}{Wasser}
  \ingredient{}{Jasminreis}
  \ingredient{nach Bedarf}{Koriander}
\begin{enumerate}
  \item Zuerst den Reis kochen, der braucht am längsten.
  \item Die Zwiebel hacken und den Karfiol waschen und in Stücke schneiden.
  \item Öl erhitzen, Knoblauch und Zwiebel anbraten, Karfiol dazu, 5 Minuten braten. Dann Curry, Kokosmilch und Wasser hinzufügen. Mit einem Deckel drauf 10 Minuten köcheln lassen.
  \item Die Kichererbsen hinzufügen und fünf Minuten ohne Deckel köcheln lassen. Am Ende mit Salz, Pfeffer und Koriander würzen.

\end{enumerate}
\end{recipe}

\end{document}
